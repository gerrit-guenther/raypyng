%% Generated by Sphinx.
\def\sphinxdocclass{report}
\documentclass[letterpaper,10pt,english]{sphinxmanual}
\ifdefined\pdfpxdimen
   \let\sphinxpxdimen\pdfpxdimen\else\newdimen\sphinxpxdimen
\fi \sphinxpxdimen=.75bp\relax
\ifdefined\pdfimageresolution
    \pdfimageresolution= \numexpr \dimexpr1in\relax/\sphinxpxdimen\relax
\fi
%% let collapsible pdf bookmarks panel have high depth per default
\PassOptionsToPackage{bookmarksdepth=5}{hyperref}

\PassOptionsToPackage{warn}{textcomp}
\usepackage[utf8]{inputenc}
\ifdefined\DeclareUnicodeCharacter
% support both utf8 and utf8x syntaxes
  \ifdefined\DeclareUnicodeCharacterAsOptional
    \def\sphinxDUC#1{\DeclareUnicodeCharacter{"#1}}
  \else
    \let\sphinxDUC\DeclareUnicodeCharacter
  \fi
  \sphinxDUC{00A0}{\nobreakspace}
  \sphinxDUC{2500}{\sphinxunichar{2500}}
  \sphinxDUC{2502}{\sphinxunichar{2502}}
  \sphinxDUC{2514}{\sphinxunichar{2514}}
  \sphinxDUC{251C}{\sphinxunichar{251C}}
  \sphinxDUC{2572}{\textbackslash}
\fi
\usepackage{cmap}
\usepackage[T1]{fontenc}
\usepackage{amsmath,amssymb,amstext}
\usepackage{babel}



\usepackage{tgtermes}
\usepackage{tgheros}
\renewcommand{\ttdefault}{txtt}



\usepackage[Bjarne]{fncychap}
\usepackage{sphinx}

\fvset{fontsize=auto}
\usepackage{geometry}


% Include hyperref last.
\usepackage{hyperref}
% Fix anchor placement for figures with captions.
\usepackage{hypcap}% it must be loaded after hyperref.
% Set up styles of URL: it should be placed after hyperref.
\urlstyle{same}

\addto\captionsenglish{\renewcommand{\contentsname}{Contents:}}

\usepackage{sphinxmessages}
\setcounter{tocdepth}{1}



\title{RayPyNG}
\date{Sep 23, 2022}
\release{}
\author{Simone Vadilonga, Ruslan Ovsyannikov}
\newcommand{\sphinxlogo}{\vbox{}}
\renewcommand{\releasename}{}
\makeindex
\begin{document}

\ifdefined\shorthandoff
  \ifnum\catcode`\=\string=\active\shorthandoff{=}\fi
  \ifnum\catcode`\"=\active\shorthandoff{"}\fi
\fi

\pagestyle{empty}
\sphinxmaketitle
\pagestyle{plain}
\sphinxtableofcontents
\pagestyle{normal}
\phantomsection\label{\detokenize{index::doc}}



\chapter{Simulate}
\label{\detokenize{index:simulate}}\index{Simulate (class in raypyng.simulate)@\spxentry{Simulate}\spxextra{class in raypyng.simulate}}

\begin{fulllineitems}
\phantomsection\label{\detokenize{index:raypyng.simulate.Simulate}}
\pysigstartsignatures
\pysiglinewithargsret{\sphinxbfcode{\sphinxupquote{class\DUrole{w}{  }}}\sphinxcode{\sphinxupquote{raypyng.simulate.}}\sphinxbfcode{\sphinxupquote{Simulate}}}{\emph{\DUrole{n}{rml}\DUrole{o}{=}\DUrole{default_value}{None}}, \emph{\DUrole{n}{hide}\DUrole{o}{=}\DUrole{default_value}{False}}, \emph{\DUrole{o}{**}\DUrole{n}{kwargs}}}{}
\pysigstopsignatures
\sphinxAtStartPar
A class that takes care of performing the simulations with RAY\sphinxhyphen{}UI
\index{\_\_init\_\_() (raypyng.simulate.Simulate method)@\spxentry{\_\_init\_\_()}\spxextra{raypyng.simulate.Simulate method}}

\begin{fulllineitems}
\phantomsection\label{\detokenize{index:raypyng.simulate.Simulate.__init__}}
\pysigstartsignatures
\pysiglinewithargsret{\sphinxbfcode{\sphinxupquote{\_\_init\_\_}}}{\emph{\DUrole{n}{rml}\DUrole{o}{=}\DUrole{default_value}{None}}, \emph{\DUrole{n}{hide}\DUrole{o}{=}\DUrole{default_value}{False}}, \emph{\DUrole{o}{**}\DUrole{n}{kwargs}}}{{ $\rightarrow$ None}}
\pysigstopsignatures
\sphinxAtStartPar
Initialize the class with a rml file
:param rml: string pointing to an rml file with the beamline template, or an RMLFile class object. Defaults to None.
:type rml: RMLFile/string, optional
:param hide: force hiding of GUI leftovers, xvfb needs to be installed. Defaults to False.
:type hide: bool, optional
\begin{quote}\begin{description}
\sphinxlineitem{Raises}
\sphinxAtStartPar
\sphinxstyleliteralstrong{\sphinxupquote{Exception}} \textendash{} If the rml file is not defined an exception is raised

\end{description}\end{quote}

\end{fulllineitems}

\index{\_\_weakref\_\_ (raypyng.simulate.Simulate attribute)@\spxentry{\_\_weakref\_\_}\spxextra{raypyng.simulate.Simulate attribute}}

\begin{fulllineitems}
\phantomsection\label{\detokenize{index:raypyng.simulate.Simulate.__weakref__}}
\pysigstartsignatures
\pysigline{\sphinxbfcode{\sphinxupquote{\_\_weakref\_\_}}}
\pysigstopsignatures
\sphinxAtStartPar
list of weak references to the object (if defined)

\end{fulllineitems}

\index{analyze (raypyng.simulate.Simulate property)@\spxentry{analyze}\spxextra{raypyng.simulate.Simulate property}}

\begin{fulllineitems}
\phantomsection\label{\detokenize{index:raypyng.simulate.Simulate.analyze}}
\pysigstartsignatures
\pysigline{\sphinxbfcode{\sphinxupquote{property\DUrole{w}{  }}}\sphinxbfcode{\sphinxupquote{analyze}}}
\pysigstopsignatures
\sphinxAtStartPar
Turn on or off the RAY\sphinxhyphen{}UI analysis of the results.
The analysis of the results takes time, so turn it on only if needed
\begin{quote}\begin{description}
\sphinxlineitem{Returns}
\sphinxAtStartPar
True: analysis on, False: analysis off

\sphinxlineitem{Return type}
\sphinxAtStartPar
bool

\end{description}\end{quote}

\end{fulllineitems}

\index{exports (raypyng.simulate.Simulate property)@\spxentry{exports}\spxextra{raypyng.simulate.Simulate property}}

\begin{fulllineitems}
\phantomsection\label{\detokenize{index:raypyng.simulate.Simulate.exports}}
\pysigstartsignatures
\pysigline{\sphinxbfcode{\sphinxupquote{property\DUrole{w}{  }}}\sphinxbfcode{\sphinxupquote{exports}}}
\pysigstopsignatures
\sphinxAtStartPar
The files to export once the simulation is complete.
for a list of possible files check self.possible\_exports
and self.possible\_exports\_without\_analysis.

\sphinxAtStartPar
It is expeceted a list of dictionaries, and for each dictionary the key is the element
to be exported and the valuee are the files to be exported

\end{fulllineitems}

\index{params (raypyng.simulate.Simulate property)@\spxentry{params}\spxextra{raypyng.simulate.Simulate property}}

\begin{fulllineitems}
\phantomsection\label{\detokenize{index:raypyng.simulate.Simulate.params}}
\pysigstartsignatures
\pysigline{\sphinxbfcode{\sphinxupquote{property\DUrole{w}{  }}}\sphinxbfcode{\sphinxupquote{params}}}
\pysigstopsignatures
\sphinxAtStartPar
The parameters to scan, as a list of dictionaries.
For each dictionary the keys are the parameters elements of the beamline, and the values are the
values to be assigned.

\end{fulllineitems}

\index{path (raypyng.simulate.Simulate property)@\spxentry{path}\spxextra{raypyng.simulate.Simulate property}}

\begin{fulllineitems}
\phantomsection\label{\detokenize{index:raypyng.simulate.Simulate.path}}
\pysigstartsignatures
\pysigline{\sphinxbfcode{\sphinxupquote{property\DUrole{w}{  }}}\sphinxbfcode{\sphinxupquote{path}}}
\pysigstopsignatures
\sphinxAtStartPar
The path where to execute the simlations
\begin{quote}\begin{description}
\sphinxlineitem{Returns}
\sphinxAtStartPar
by default the path is the current path from which
the program is executed

\sphinxlineitem{Return type}
\sphinxAtStartPar
string

\end{description}\end{quote}

\end{fulllineitems}

\index{possible\_exports (raypyng.simulate.Simulate property)@\spxentry{possible\_exports}\spxextra{raypyng.simulate.Simulate property}}

\begin{fulllineitems}
\phantomsection\label{\detokenize{index:raypyng.simulate.Simulate.possible_exports}}
\pysigstartsignatures
\pysigline{\sphinxbfcode{\sphinxupquote{property\DUrole{w}{  }}}\sphinxbfcode{\sphinxupquote{possible\_exports}}}
\pysigstopsignatures
\sphinxAtStartPar
A list of the files that can be exported by RAY\sphinxhyphen{}UI
\begin{quote}\begin{description}
\sphinxlineitem{Returns}
\sphinxAtStartPar
list of the names of the possible exports for RAY\sphinxhyphen{}UI

\sphinxlineitem{Return type}
\sphinxAtStartPar
list

\end{description}\end{quote}

\end{fulllineitems}

\index{possible\_exports\_without\_analysis (raypyng.simulate.Simulate property)@\spxentry{possible\_exports\_without\_analysis}\spxextra{raypyng.simulate.Simulate property}}

\begin{fulllineitems}
\phantomsection\label{\detokenize{index:raypyng.simulate.Simulate.possible_exports_without_analysis}}
\pysigstartsignatures
\pysigline{\sphinxbfcode{\sphinxupquote{property\DUrole{w}{  }}}\sphinxbfcode{\sphinxupquote{possible\_exports\_without\_analysis}}}
\pysigstopsignatures
\sphinxAtStartPar
A list of the files that can be exported by RAY\sphinxhyphen{}UI when the
analysis option is turned off
\begin{quote}\begin{description}
\sphinxlineitem{Returns}
\sphinxAtStartPar
list of the names of the possible exports for RAY\sphinxhyphen{}UI when analysis is off

\sphinxlineitem{Return type}
\sphinxAtStartPar
list

\end{description}\end{quote}

\end{fulllineitems}

\index{repeat (raypyng.simulate.Simulate property)@\spxentry{repeat}\spxextra{raypyng.simulate.Simulate property}}

\begin{fulllineitems}
\phantomsection\label{\detokenize{index:raypyng.simulate.Simulate.repeat}}
\pysigstartsignatures
\pysigline{\sphinxbfcode{\sphinxupquote{property\DUrole{w}{  }}}\sphinxbfcode{\sphinxupquote{repeat}}}
\pysigstopsignatures
\sphinxAtStartPar
The simulations can be repeated an arbitrary number of times
If the statitcs are not good enough using 2 millions of rays is suggested
to repeat them instead of increasing the number of rays
\begin{quote}\begin{description}
\sphinxlineitem{Returns}
\sphinxAtStartPar
the number of repetition of the simulations, by default is 1

\sphinxlineitem{Return type}
\sphinxAtStartPar
int

\end{description}\end{quote}

\end{fulllineitems}

\index{rml (raypyng.simulate.Simulate property)@\spxentry{rml}\spxextra{raypyng.simulate.Simulate property}}

\begin{fulllineitems}
\phantomsection\label{\detokenize{index:raypyng.simulate.Simulate.rml}}
\pysigstartsignatures
\pysigline{\sphinxbfcode{\sphinxupquote{property\DUrole{w}{  }}}\sphinxbfcode{\sphinxupquote{rml}}}
\pysigstopsignatures
\sphinxAtStartPar
RMLFile object instantiated in init

\end{fulllineitems}

\index{rml\_list() (raypyng.simulate.Simulate method)@\spxentry{rml\_list()}\spxextra{raypyng.simulate.Simulate method}}

\begin{fulllineitems}
\phantomsection\label{\detokenize{index:raypyng.simulate.Simulate.rml_list}}
\pysigstartsignatures
\pysiglinewithargsret{\sphinxbfcode{\sphinxupquote{rml\_list}}}{}{}
\pysigstopsignatures
\sphinxAtStartPar
This function creates the folder structure and the rml files to simulate.
It requires the param to be set. Useful if one wants to create the simulation files
for a manual check before starting the simulations.

\end{fulllineitems}

\index{run() (raypyng.simulate.Simulate method)@\spxentry{run()}\spxextra{raypyng.simulate.Simulate method}}

\begin{fulllineitems}
\phantomsection\label{\detokenize{index:raypyng.simulate.Simulate.run}}
\pysigstartsignatures
\pysiglinewithargsret{\sphinxbfcode{\sphinxupquote{run}}}{\emph{\DUrole{n}{recipe}\DUrole{o}{=}\DUrole{default_value}{None}}, \emph{\DUrole{o}{/}}, \emph{\DUrole{n}{multiprocessing}\DUrole{o}{=}\DUrole{default_value}{True}}, \emph{\DUrole{n}{force}\DUrole{o}{=}\DUrole{default_value}{False}}}{}
\pysigstopsignatures
\sphinxAtStartPar
This method starts the simulations. params and exports need to be defined.
\begin{quote}\begin{description}
\sphinxlineitem{Parameters}\begin{itemize}
\item {} 
\sphinxAtStartPar
\sphinxstyleliteralstrong{\sphinxupquote{recipe}} (\sphinxstyleliteralemphasis{\sphinxupquote{SimulationRecipe}}\sphinxstyleliteralemphasis{\sphinxupquote{, }}\sphinxstyleliteralemphasis{\sphinxupquote{optional}}) \textendash{} If using a recipee pass it as a parameter. Defaults to None.

\item {} 
\sphinxAtStartPar
\sphinxstyleliteralstrong{\sphinxupquote{multiprocessing}} (\sphinxstyleliteralemphasis{\sphinxupquote{boolint}}\sphinxstyleliteralemphasis{\sphinxupquote{, }}\sphinxstyleliteralemphasis{\sphinxupquote{optional}}) \textendash{} If True all the cpus are used. If an integer n is provided, n cpus are used. Defaults to True.

\item {} 
\sphinxAtStartPar
\sphinxstyleliteralstrong{\sphinxupquote{force}} (\sphinxstyleliteralemphasis{\sphinxupquote{bool}}\sphinxstyleliteralemphasis{\sphinxupquote{, }}\sphinxstyleliteralemphasis{\sphinxupquote{optional}}) \textendash{} If True all the simlations are performed, even if the export files already exist. If False only the simlations for which are missing some exports are performed. Defaults to False.

\end{itemize}

\end{description}\end{quote}

\end{fulllineitems}

\index{simulation\_name (raypyng.simulate.Simulate property)@\spxentry{simulation\_name}\spxextra{raypyng.simulate.Simulate property}}

\begin{fulllineitems}
\phantomsection\label{\detokenize{index:raypyng.simulate.Simulate.simulation_name}}
\pysigstartsignatures
\pysigline{\sphinxbfcode{\sphinxupquote{property\DUrole{w}{  }}}\sphinxbfcode{\sphinxupquote{simulation\_name}}}
\pysigstopsignatures
\sphinxAtStartPar
A string to append to the folder where the simulations will be executed.

\end{fulllineitems}


\end{fulllineitems}



\chapter{SimulationParams}
\label{\detokenize{index:simulationparams}}\index{SimulationParams (class in raypyng.simulate)@\spxentry{SimulationParams}\spxextra{class in raypyng.simulate}}

\begin{fulllineitems}
\phantomsection\label{\detokenize{index:raypyng.simulate.SimulationParams}}
\pysigstartsignatures
\pysiglinewithargsret{\sphinxbfcode{\sphinxupquote{class\DUrole{w}{  }}}\sphinxcode{\sphinxupquote{raypyng.simulate.}}\sphinxbfcode{\sphinxupquote{SimulationParams}}}{\emph{\DUrole{n}{rml}\DUrole{o}{=}\DUrole{default_value}{None}}, \emph{\DUrole{n}{param\_list}\DUrole{o}{=}\DUrole{default_value}{None}}, \emph{\DUrole{o}{**}\DUrole{n}{kwargs}}}{}
\pysigstopsignatures
\sphinxAtStartPar
A class that takes care of the simulations parameters, makes sure that they are written correctly,
and returns the the list of simulations that is requested by the user.
\index{\_\_init\_\_() (raypyng.simulate.SimulationParams method)@\spxentry{\_\_init\_\_()}\spxextra{raypyng.simulate.SimulationParams method}}

\begin{fulllineitems}
\phantomsection\label{\detokenize{index:raypyng.simulate.SimulationParams.__init__}}
\pysigstartsignatures
\pysiglinewithargsret{\sphinxbfcode{\sphinxupquote{\_\_init\_\_}}}{\emph{\DUrole{n}{rml}\DUrole{o}{=}\DUrole{default_value}{None}}, \emph{\DUrole{n}{param\_list}\DUrole{o}{=}\DUrole{default_value}{None}}, \emph{\DUrole{o}{**}\DUrole{n}{kwargs}}}{{ $\rightarrow$ None}}
\pysigstopsignatures
\sphinxAtStartPar
\_summary\_
\begin{quote}\begin{description}
\sphinxlineitem{Parameters}\begin{itemize}
\item {} 
\sphinxAtStartPar
\sphinxstyleliteralstrong{\sphinxupquote{rml}} (\sphinxstyleliteralemphasis{\sphinxupquote{RMLFile/string}}\sphinxstyleliteralemphasis{\sphinxupquote{, }}\sphinxstyleliteralemphasis{\sphinxupquote{optional}}) \textendash{} string pointing to an rml file with the beamline template, or an RMLFile class object. Defaults to None.

\item {} 
\sphinxAtStartPar
\sphinxstyleliteralstrong{\sphinxupquote{param\_list}} (\sphinxstyleliteralemphasis{\sphinxupquote{list}}\sphinxstyleliteralemphasis{\sphinxupquote{, }}\sphinxstyleliteralemphasis{\sphinxupquote{optional}}) \textendash{} list of dictionaries containing the parameters and values to simulate. Defaults to None.

\end{itemize}

\end{description}\end{quote}

\end{fulllineitems}

\index{\_\_weakref\_\_ (raypyng.simulate.SimulationParams attribute)@\spxentry{\_\_weakref\_\_}\spxextra{raypyng.simulate.SimulationParams attribute}}

\begin{fulllineitems}
\phantomsection\label{\detokenize{index:raypyng.simulate.SimulationParams.__weakref__}}
\pysigstartsignatures
\pysigline{\sphinxbfcode{\sphinxupquote{\_\_weakref\_\_}}}
\pysigstopsignatures
\sphinxAtStartPar
list of weak references to the object (if defined)

\end{fulllineitems}

\index{\_calc\_loop() (raypyng.simulate.SimulationParams method)@\spxentry{\_calc\_loop()}\spxextra{raypyng.simulate.SimulationParams method}}

\begin{fulllineitems}
\phantomsection\label{\detokenize{index:raypyng.simulate.SimulationParams._calc_loop}}
\pysigstartsignatures
\pysiglinewithargsret{\sphinxbfcode{\sphinxupquote{\_calc\_loop}}}{\emph{\DUrole{n}{verbose}\DUrole{p}{:}\DUrole{w}{  }\DUrole{n}{bool}\DUrole{w}{  }\DUrole{o}{=}\DUrole{w}{  }\DUrole{default_value}{True}}}{}
\pysigstopsignatures
\sphinxAtStartPar
Calculate the simulations loop
\begin{quote}\begin{description}
\sphinxlineitem{Returns}
\sphinxAtStartPar
idependent and dependent parameters
self.simulations\_param\_list (list): parameters values for each simulation loop

\sphinxlineitem{Return type}
\sphinxAtStartPar
self.param\_to\_simulate (list)

\end{description}\end{quote}

\end{fulllineitems}

\index{\_check\_if\_enabled() (raypyng.simulate.SimulationParams method)@\spxentry{\_check\_if\_enabled()}\spxextra{raypyng.simulate.SimulationParams method}}

\begin{fulllineitems}
\phantomsection\label{\detokenize{index:raypyng.simulate.SimulationParams._check_if_enabled}}
\pysigstartsignatures
\pysiglinewithargsret{\sphinxbfcode{\sphinxupquote{\_check\_if\_enabled}}}{\emph{\DUrole{n}{param}}}{}
\pysigstopsignatures
\sphinxAtStartPar
Check if a parameter is enabled
\begin{quote}\begin{description}
\sphinxlineitem{Parameters}
\sphinxAtStartPar
\sphinxstyleliteralstrong{\sphinxupquote{param}} (\sphinxstyleliteralemphasis{\sphinxupquote{RML object}}) \textendash{} an parameter to simulate

\sphinxlineitem{Returns}
\sphinxAtStartPar
True if the parameter is enabled, False otherwise

\sphinxlineitem{Return type}
\sphinxAtStartPar
(bool)

\end{description}\end{quote}

\end{fulllineitems}

\index{\_check\_param() (raypyng.simulate.SimulationParams method)@\spxentry{\_check\_param()}\spxextra{raypyng.simulate.SimulationParams method}}

\begin{fulllineitems}
\phantomsection\label{\detokenize{index:raypyng.simulate.SimulationParams._check_param}}
\pysigstartsignatures
\pysiglinewithargsret{\sphinxbfcode{\sphinxupquote{\_check\_param}}}{}{}
\pysigstopsignatures
\sphinxAtStartPar
Check that self.param is a list of dictionaries, and convert the
items of the dictionaries to lists, otherwise raise an exception.

\end{fulllineitems}

\index{\_enable\_param() (raypyng.simulate.SimulationParams method)@\spxentry{\_enable\_param()}\spxextra{raypyng.simulate.SimulationParams method}}

\begin{fulllineitems}
\phantomsection\label{\detokenize{index:raypyng.simulate.SimulationParams._enable_param}}
\pysigstartsignatures
\pysiglinewithargsret{\sphinxbfcode{\sphinxupquote{\_enable\_param}}}{\emph{\DUrole{n}{param}}}{}
\pysigstopsignatures
\sphinxAtStartPar
Set enabled to True in a beamline object, and auto to False
\begin{quote}\begin{description}
\sphinxlineitem{Parameters}
\sphinxAtStartPar
\sphinxstyleliteralstrong{\sphinxupquote{param}} (\sphinxstyleliteralemphasis{\sphinxupquote{RML object}}) \textendash{} beamline object

\end{description}\end{quote}

\end{fulllineitems}

\index{\_extract\_param() (raypyng.simulate.SimulationParams method)@\spxentry{\_extract\_param()}\spxextra{raypyng.simulate.SimulationParams method}}

\begin{fulllineitems}
\phantomsection\label{\detokenize{index:raypyng.simulate.SimulationParams._extract_param}}
\pysigstartsignatures
\pysiglinewithargsret{\sphinxbfcode{\sphinxupquote{\_extract\_param}}}{\emph{\DUrole{n}{verbose}\DUrole{p}{:}\DUrole{w}{  }\DUrole{n}{bool}\DUrole{w}{  }\DUrole{o}{=}\DUrole{w}{  }\DUrole{default_value}{False}}}{}
\pysigstopsignatures
\sphinxAtStartPar
Parse self.param and extract dependent and independent parameters
\begin{quote}\begin{description}
\sphinxlineitem{Parameters}
\sphinxAtStartPar
\sphinxstyleliteralstrong{\sphinxupquote{verbose}} (\sphinxstyleliteralemphasis{\sphinxupquote{bool}}\sphinxstyleliteralemphasis{\sphinxupquote{, }}\sphinxstyleliteralemphasis{\sphinxupquote{optional}}) \textendash{} If True print the returned objects. Defaults to False.

\sphinxlineitem{Returns}
\sphinxAtStartPar
indieendent parameter values
self.ind\_par (list): independent parameters
self.dep\_param\_dependency (dict): dictionary of dependencies
self.dep\_value\_dependency (list): dictionaries of dependent values
self.dep\_par (list): dependent parameters

\sphinxlineitem{Return type}
\sphinxAtStartPar
self.ind\_param\_values (list)

\end{description}\end{quote}

\end{fulllineitems}

\index{\_write\_value\_to\_param() (raypyng.simulate.SimulationParams method)@\spxentry{\_write\_value\_to\_param()}\spxextra{raypyng.simulate.SimulationParams method}}

\begin{fulllineitems}
\phantomsection\label{\detokenize{index:raypyng.simulate.SimulationParams._write_value_to_param}}
\pysigstartsignatures
\pysiglinewithargsret{\sphinxbfcode{\sphinxupquote{\_write\_value\_to\_param}}}{\emph{\DUrole{n}{param}}, \emph{\DUrole{n}{value}}}{}
\pysigstopsignatures
\sphinxAtStartPar
Write a value to a parameter, making sure enable is T
and auto is F
\begin{quote}\begin{description}
\sphinxlineitem{Parameters}\begin{itemize}
\item {} 
\sphinxAtStartPar
\sphinxstyleliteralstrong{\sphinxupquote{param}} (\sphinxstyleliteralemphasis{\sphinxupquote{RML object}}) \textendash{} beamline object

\item {} 
\sphinxAtStartPar
\sphinxstyleliteralstrong{\sphinxupquote{value}} (\sphinxstyleliteralemphasis{\sphinxupquote{str}}\sphinxstyleliteralemphasis{\sphinxupquote{,}}\sphinxstyleliteralemphasis{\sphinxupquote{int}}\sphinxstyleliteralemphasis{\sphinxupquote{,}}\sphinxstyleliteralemphasis{\sphinxupquote{float}}) \textendash{} the value to set the beamline object to

\end{itemize}

\end{description}\end{quote}

\end{fulllineitems}

\index{params (raypyng.simulate.SimulationParams property)@\spxentry{params}\spxextra{raypyng.simulate.SimulationParams property}}

\begin{fulllineitems}
\phantomsection\label{\detokenize{index:raypyng.simulate.SimulationParams.params}}
\pysigstartsignatures
\pysigline{\sphinxbfcode{\sphinxupquote{property\DUrole{w}{  }}}\sphinxbfcode{\sphinxupquote{params}}}
\pysigstopsignatures
\sphinxAtStartPar
The parameters to scan, as a list of dictionaries.
For each dictionary the keys are the parameters elements of the beamline, and the values are the
values to be assigned.

\end{fulllineitems}

\index{rml (raypyng.simulate.SimulationParams property)@\spxentry{rml}\spxextra{raypyng.simulate.SimulationParams property}}

\begin{fulllineitems}
\phantomsection\label{\detokenize{index:raypyng.simulate.SimulationParams.rml}}
\pysigstartsignatures
\pysigline{\sphinxbfcode{\sphinxupquote{property\DUrole{w}{  }}}\sphinxbfcode{\sphinxupquote{rml}}}
\pysigstopsignatures
\sphinxAtStartPar
RMLFile object instantiated in init

\end{fulllineitems}


\end{fulllineitems}



\chapter{RayUIRunner}
\label{\detokenize{index:rayuirunner}}\index{RayUIRunner (class in raypyng.runner)@\spxentry{RayUIRunner}\spxextra{class in raypyng.runner}}

\begin{fulllineitems}
\phantomsection\label{\detokenize{index:raypyng.runner.RayUIRunner}}
\pysigstartsignatures
\pysiglinewithargsret{\sphinxbfcode{\sphinxupquote{class\DUrole{w}{  }}}\sphinxcode{\sphinxupquote{raypyng.runner.}}\sphinxbfcode{\sphinxupquote{RayUIRunner}}}{\emph{\DUrole{n}{ray\_path}\DUrole{o}{=}\DUrole{default_value}{None}}, \emph{\DUrole{n}{ray\_binary}\DUrole{o}{=}\DUrole{default_value}{\textquotesingle{}rayui.sh\textquotesingle{}}}, \emph{\DUrole{n}{background}\DUrole{o}{=}\DUrole{default_value}{True}}, \emph{\DUrole{n}{hide}\DUrole{o}{=}\DUrole{default_value}{False}}}{}
\pysigstopsignatures
\sphinxAtStartPar
RayUIRunner class implements all logic to start a RayUI process
\index{\_\_detect\_ray\_path() (raypyng.runner.RayUIRunner method)@\spxentry{\_\_detect\_ray\_path()}\spxextra{raypyng.runner.RayUIRunner method}}

\begin{fulllineitems}
\phantomsection\label{\detokenize{index:raypyng.runner.RayUIRunner.__detect_ray_path}}
\pysigstartsignatures
\pysiglinewithargsret{\sphinxbfcode{\sphinxupquote{\_\_detect\_ray\_path}}}{}{{ $\rightarrow$ str}}
\pysigstopsignatures
\sphinxAtStartPar
Internal function to autodetect installation path of RayUI
\begin{quote}\begin{description}
\sphinxlineitem{Raises}
\sphinxAtStartPar
\sphinxstyleliteralstrong{\sphinxupquote{RayPyRunnerError}} \textendash{} is case no ray installations can be detected

\sphinxlineitem{Returns}
\sphinxAtStartPar
string with the detected ray installation path

\sphinxlineitem{Return type}
\sphinxAtStartPar
str

\end{description}\end{quote}

\end{fulllineitems}

\index{\_\_init\_\_() (raypyng.runner.RayUIRunner method)@\spxentry{\_\_init\_\_()}\spxextra{raypyng.runner.RayUIRunner method}}

\begin{fulllineitems}
\phantomsection\label{\detokenize{index:raypyng.runner.RayUIRunner.__init__}}
\pysigstartsignatures
\pysiglinewithargsret{\sphinxbfcode{\sphinxupquote{\_\_init\_\_}}}{\emph{\DUrole{n}{ray\_path}\DUrole{o}{=}\DUrole{default_value}{None}}, \emph{\DUrole{n}{ray\_binary}\DUrole{o}{=}\DUrole{default_value}{\textquotesingle{}rayui.sh\textquotesingle{}}}, \emph{\DUrole{n}{background}\DUrole{o}{=}\DUrole{default_value}{True}}, \emph{\DUrole{n}{hide}\DUrole{o}{=}\DUrole{default_value}{False}}}{{ $\rightarrow$ None}}
\pysigstopsignatures
\end{fulllineitems}

\index{\_\_weakref\_\_ (raypyng.runner.RayUIRunner attribute)@\spxentry{\_\_weakref\_\_}\spxextra{raypyng.runner.RayUIRunner attribute}}

\begin{fulllineitems}
\phantomsection\label{\detokenize{index:raypyng.runner.RayUIRunner.__weakref__}}
\pysigstartsignatures
\pysigline{\sphinxbfcode{\sphinxupquote{\_\_weakref\_\_}}}
\pysigstopsignatures
\sphinxAtStartPar
list of weak references to the object (if defined)

\end{fulllineitems}

\index{\_readline() (raypyng.runner.RayUIRunner method)@\spxentry{\_readline()}\spxextra{raypyng.runner.RayUIRunner method}}

\begin{fulllineitems}
\phantomsection\label{\detokenize{index:raypyng.runner.RayUIRunner._readline}}
\pysigstartsignatures
\pysiglinewithargsret{\sphinxbfcode{\sphinxupquote{\_readline}}}{}{{ $\rightarrow$ str}}
\pysigstopsignatures
\sphinxAtStartPar
read a line from the stdout of the process and convert to a string
\begin{quote}\begin{description}
\sphinxlineitem{Returns}
\sphinxAtStartPar
line read from the input

\sphinxlineitem{Return type}
\sphinxAtStartPar
str

\end{description}\end{quote}

\end{fulllineitems}

\index{\_write() (raypyng.runner.RayUIRunner method)@\spxentry{\_write()}\spxextra{raypyng.runner.RayUIRunner method}}

\begin{fulllineitems}
\phantomsection\label{\detokenize{index:raypyng.runner.RayUIRunner._write}}
\pysigstartsignatures
\pysiglinewithargsret{\sphinxbfcode{\sphinxupquote{\_write}}}{\emph{\DUrole{n}{instr}\DUrole{p}{:}\DUrole{w}{  }\DUrole{n}{str}}, \emph{\DUrole{n}{endline}\DUrole{o}{=}\DUrole{default_value}{\textquotesingle{}\textbackslash{}n\textquotesingle{}}}}{}
\pysigstopsignatures
\sphinxAtStartPar
Write command to RayUI interface
\begin{quote}\begin{description}
\sphinxlineitem{Parameters}\begin{itemize}
\item {} 
\sphinxAtStartPar
\sphinxstyleliteralstrong{\sphinxupquote{instr}} (\sphinxstyleliteralemphasis{\sphinxupquote{str}}) \textendash{} \_description\_

\item {} 
\sphinxAtStartPar
\sphinxstyleliteralstrong{\sphinxupquote{endline}} (\sphinxstyleliteralemphasis{\sphinxupquote{str}}\sphinxstyleliteralemphasis{\sphinxupquote{, }}\sphinxstyleliteralemphasis{\sphinxupquote{optional}}) \textendash{} \_description\_. Defaults to endline character.

\end{itemize}

\sphinxlineitem{Raises}
\sphinxAtStartPar
\sphinxstyleliteralstrong{\sphinxupquote{RayPyRunnerError}} \textendash{} \_description\_

\end{description}\end{quote}

\end{fulllineitems}

\index{isrunning (raypyng.runner.RayUIRunner property)@\spxentry{isrunning}\spxextra{raypyng.runner.RayUIRunner property}}

\begin{fulllineitems}
\phantomsection\label{\detokenize{index:raypyng.runner.RayUIRunner.isrunning}}
\pysigstartsignatures
\pysigline{\sphinxbfcode{\sphinxupquote{property\DUrole{w}{  }}}\sphinxbfcode{\sphinxupquote{isrunning}}}
\pysigstopsignatures
\sphinxAtStartPar
Check weather a process is running and rerutn a boolean
\begin{quote}\begin{description}
\sphinxlineitem{Returns}
\sphinxAtStartPar
returns True if the process is running, otherwise False

\sphinxlineitem{Return type}
\sphinxAtStartPar
bool

\end{description}\end{quote}

\end{fulllineitems}

\index{kill() (raypyng.runner.RayUIRunner method)@\spxentry{kill()}\spxextra{raypyng.runner.RayUIRunner method}}

\begin{fulllineitems}
\phantomsection\label{\detokenize{index:raypyng.runner.RayUIRunner.kill}}
\pysigstartsignatures
\pysiglinewithargsret{\sphinxbfcode{\sphinxupquote{kill}}}{}{}
\pysigstopsignatures
\sphinxAtStartPar
kill a RAY\sphinxhyphen{}UI process

\end{fulllineitems}

\index{pid (raypyng.runner.RayUIRunner property)@\spxentry{pid}\spxextra{raypyng.runner.RayUIRunner property}}

\begin{fulllineitems}
\phantomsection\label{\detokenize{index:raypyng.runner.RayUIRunner.pid}}
\pysigstartsignatures
\pysigline{\sphinxbfcode{\sphinxupquote{property\DUrole{w}{  }}}\sphinxbfcode{\sphinxupquote{pid}}}
\pysigstopsignatures
\sphinxAtStartPar
Get process id of the RayUI process
\begin{quote}\begin{description}
\sphinxlineitem{Returns}
\sphinxAtStartPar
PID of the process if it running, None otherwise

\sphinxlineitem{Return type}
\sphinxAtStartPar
\_type\_

\end{description}\end{quote}

\end{fulllineitems}

\index{run() (raypyng.runner.RayUIRunner method)@\spxentry{run()}\spxextra{raypyng.runner.RayUIRunner method}}

\begin{fulllineitems}
\phantomsection\label{\detokenize{index:raypyng.runner.RayUIRunner.run}}
\pysigstartsignatures
\pysiglinewithargsret{\sphinxbfcode{\sphinxupquote{run}}}{}{}
\pysigstopsignatures
\sphinxAtStartPar
Open one instance of RAY\sphinxhyphen{}UI using subprocess
\begin{quote}\begin{description}
\sphinxlineitem{Raises}
\sphinxAtStartPar
\sphinxstyleliteralstrong{\sphinxupquote{RayPyRunnerError}} \textendash{} if the RAY\sphinxhyphen{}UI executable is not found raise an error

\end{description}\end{quote}

\end{fulllineitems}


\end{fulllineitems}




\renewcommand{\indexname}{Index}
\printindex
\end{document}